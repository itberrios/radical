\documentclass{article}




%%%%%%%%%%%%%%%%%%%%%% Formatting Stuff %%%%%%%%%%%%%%%%%%%%%%%%%%%
\usepackage{times}
\usepackage[T1]{fontenc}

\setlength{\parskip}{1em}\setlength{\parindent}{0pt}
\linespread{1.25}
\usepackage[margin=0.7in,top=1in]{geometry}\usepackage{fancyhdr}
\newcommand{\info}{\clearpage \subsection*{Information}}
\newcommand{\solution}[1]{\clearpage \subsection*{Solution #1}}
\newcommand{\spart}[1]{\paragraph{(#1)}}
\newcommand{\sspart}[1]{\subparagraph{(#1)}}

% increase vertical spacing for matrices
%\makeatletter
%\renewcommand*\env@matrix[1][\arraystretch]{%
%  \edef\arraystretch{#1}%
%  \hskip -\arraycolsep
%  \let\@ifnextchar\new@ifnextchar
%  \array{*\c@MaxMatrixCols c}}
%\makeatother


%%%%%%%%%%%%%%%%%%%%%%%%%%%%%%%%%%%%%%%%%%%%%%%%%%%%%%%%%%%%%%%%%%%


%%% Add any more packages if you want to
\usepackage{amsmath}
% \usepackage{physics}
\usepackage{mathtools}
\usepackage{amsfonts}
\usepackage{outlines}
\usepackage{subcaption}
\usepackage{hyperref}
\usepackage{nameref}
\usepackage{amsthm}
\usepackage{amsfonts}
\usepackage{amssymb}
\usepackage{environ}
\usepackage{graphicx} 
\usepackage{enumitem}
\usepackage{scalerel}
\usepackage{bm}
\usepackage{csvsimple}
\setlist[description]{leftmargin=\parindent,labelindent=\parindent}

% custom definitions
% New command to make a transpose symbol.
% The normal method places the Transpose a bit too low, so this % method places it a bit higher.

\makeatletter
\newcommand*{\transpose}{%
  {\mathpalette\@transpose{}}%
}
\newcommand*{\@transpose}[2]{%
  % #1: math style
  % #2: unused
  \raisebox{\depth}{$\m@th#1\intercal$}%
}

\newcommand{\problem}[1]{\clearpage \subsection*{\textbf{Problem #1}}}


% new commands for matrix formatting
% \newcommand*{\matr}[1]{\mathbfit{#1}}
\newcommand*{\matr}[1]{\mathbf{#1}}
\newcommand*{\tran}{^{\mkern-1.5mu\mathsf{T}}}
\newcommand*{\conj}[1]{\overline{#1}}
\newcommand*{\hermconj}{^{\mathsf{H}}}

% new commands for probability notations
\newcommand\iid{$i.i.d.$}
\newcommand\pN{\mathcal{N}}
% \iid~$X \sim \pN(\mu, \sigma^2)$

% misc math commands
\DeclareMathOperator*{\argmin}{argmin}
\DeclareMathOperator*{\argmax}{argmax} 



\begin{document}


%%%%% Main Body goes here
% Example

% \problem{1}
% \spart{a}


% \begin{figure*}[!h]
%   \centering
%   {\includegraphics[scale=0.35]{grid_rewards_4x4.png}\label{fig:f1}}
%   \caption{Initial 4x4 Grid Rewards}
% \end{figure*}

% \newpage


% \begin{itemize}[topsep=2pt,itemsep=2pt,partopsep=2pt, parsep=2pt]
% 	\item $\epsilon = 1e-6$ 
% 	\item Stopping Criteria: if $max\lbrace 0, \sum v_{\pi_t} - v_{\pi_{t-1}} \rbrace < \epsilon$, then break
% 	\item $\gamma = 0.7$
% 	\item Initialize $v_{\pi} = 0_{4 x 4}$
% 	\item Initialize $Pi$ as an empty 4x4 matrix
% \end{itemize}

\section{Snapshot Model}

\begin{align*}
	x(t) = v(\theta) f(t) + n(t)
	\\
	\\
	\\
	X = V F + N
	\\
	\\
	\\
	X = X_s + N
\end{align*}


\begin{itemize}
	\item x - (Nx1) \text{ Received signal vector at time } t
	
	\item v - (NxD) \text{ Matrix of Steering Vectors }
	
	\item f(t) - (Dx1) \text{ Zero Mean Random Vector that contains desired } \\ \text{ and possibly undesired signals (D signals in total) }

	\item n(t) - (Nx1) \text{ Complex Additive White Gaussian Noise }
	
\end{itemize}

\begin{itemize}
	\item X - (NxT) \text{ Full Signal as received by the Array }
	
	\item V - (NxD) \text{ Matrix of Steering Vectors }
	
	\item F - (DxT) \text{ Zero Mean Random Vector that contains desired } \\ \text{ and possibly undesired signals (D signals in total) }

	\item N - (NxT) \text{ Complex Additive White Gaussian Noise }
	
\end{itemize}


\newpage


\section{Bartlett Beamforming}

\begin{align*}
	y = w^\mathsf{H} X
	\\
	\\
	y = v^\mathsf{H}(\theta_s) X
	\\
	f = y = v^\mathsf{H}(\theta_s) X
	\\
	\mathcal{P} = \frac{1}{T} \sum_{t=0}^{T-1} | v^\mathsf{H}(\theta_s) x(t) | ^2
	\\
	\\
\end{align*}


\begin{align*}
	\mathcal{P} &= Var(f)
	\\
	&= E[(f - \mu_f)^2]
	\\
	&= E[f^2] \qquad\qquad\qquad\qquad \text{ zero mean signal }
	\\
	&= E[|v^\mathsf{H}(\theta_s) X|^2] \qquad\qquad\, \text{ ideal Beamformer }
	\\
	&= \frac{1}{T} \sum_{t=0}^{T-1} |v^\mathsf{H}(\theta_s) x(t)|^2
\end{align*}



\begin{align*}
	\mathcal{P} &= \frac{1}{T} \sum_{t=0}^{T-1} | v^\mathsf{H}(\theta) x(t) | ^2
	\\
	&= \frac{1}{T} \sum_{t=0}^{T-1} ( v^\mathsf{H} x(t))(v^\mathsf{H} x(t))^\mathsf{H}
	\\
	&= \frac{1}{T} \sum_{t=0}^{T-1}  v^\mathsf{H} x(t) x^\mathsf{H}(t) v_s
	\\
	&= v^\mathsf{H} \left( \frac{1}{T} \sum_{t=0}^{T-1} x(t) x^\mathsf{H}(t) \right) v 
	\\
	&= v^\mathsf{H} E[X X^\mathsf{H}] v 
	\\
	\mathcal{P}_B &= v^\mathsf{H} R_{XX} v
	\\
\end{align*}

\newpage


\section{Capon Beamformer}

\begin{align*}
	\hat{y} = w^\mathsf{H} X
	\\
	\hat{y} = f
	\\
	v^\mathsf{H}(\theta_s) w = 1
	\\
	f = y = v^\mathsf{H}(\theta_s) X
	\\
	\mathcal{P} = \frac{1}{T} \sum_{t=0}^{T-1} | v^\mathsf{H}(\theta_s) x(t) | ^2
	\\
	\\
\end{align*}

\begin{align*}
	\hat{y} &= w^\mathsf{H} X
	\\
	&= w^\mathsf{H} (X_s + N)
	\\
	&= w^\mathsf{H} X_s + w^\mathsf{H} N
	\\
	&= f + w^\mathsf{H} N \qquad\qquad \text{ Distortionaless Response }
	\\
	\hat{y} &= f + Y_n
\end{align*}


\begin{align*}
	Var(\hat{y}) &= E[|\hat{y}|^2] + E[\hat{y}]^2
	\\
	&= E[|\hat{y}|^2] + E[f + Y_n]^2
	\\
	&= E[|\hat{y}|^2] + (E[f] + E[Y_n])^2 \qquad\qquad \text{ both $f$ and $Y_n$ are zero mean }
	\\
	&= E[|\hat{y}|^2] + 0
	\\
	&= E[|w^\mathsf{H} X|^2]
	\\
	&= E[(w^\mathsf{H} X)(w^\mathsf{H} X)^\mathsf{H}]
	\\
	&= E[w^\mathsf{H} X X^\mathsf{H} w]
	\\
	&= w^\mathsf{H} E[X X^\mathsf{H}] w
	\\
	&= w^\mathsf{H} \left( \frac{1}{T} \sum_{t=0}^{T-1} x(t) x(t)^\mathsf{H} \right) w
	\\
	&= w^\mathsf{H} R_{XX} w
\end{align*}


\begin{align*}
	\text{ Minimize } w^\mathsf{H} R_{XX} w \\
	\text{subject to} : w^\mathsf{H}v_s  = 1
	\\
	\\
	\text{first construct the objective function}
	\\
	\mathcal{L} = w^\mathsf{H} R_{XX} w + \text{Re} [\lambda (w^\mathsf{H}v_s - 1)]
\end{align*}


\begin{align*}
	\mathcal{L} &= w^\mathsf{H} R_{XX} w + \text{Re} [\lambda (w^\mathsf{H}v_s - 1)]
	\\
	&= w^\mathsf{H} R_{XX} w + \frac{\lambda}{2} (w^\mathsf{H}v_s - 1) + \left[ \frac{\lambda}{2} (w^\mathsf{H}v_s - 1) \right]^*
	\\
	&= w^\mathsf{H} R_{XX} w + \frac{\lambda}{2} (w^\mathsf{H}v_s - 1) + \frac{\lambda}{2}^* (w^\mathsf{T}v_s^* - 1)
	\\
	&= w^\mathsf{H} R_{XX} w + \frac{\lambda}{2} (w^\mathsf{H}v_s - 1) + \frac{\lambda}{2}^* (v_s^\mathsf{H} w - 1)
\end{align*}


\begin{align*}
	\mathcal{L} = w^\mathsf{H} R_{XX} w + \frac{\lambda}{2} (w^\mathsf{H}v_s - 1) + &\frac{\lambda}{2}^* (v_s^\mathsf{H} w - 1)
	\\
	\frac{\partial \mathcal{L}}{\partial w^\mathsf{H}} = R_{XX} w + \frac{\lambda}{2} v_s &= 0
	\\
	R_{XX} w &= - \frac{\lambda}{2} v_s 
	\\
	w &= - \frac{\lambda}{2} R_{XX}^{-1} v_s 
	\\
\end{align*}


\begin{align*}
	w^\mathsf{H}v_s = v_s^\mathsf{H}w &= 1
	\\
	v_s^\mathsf{H}w = - \frac{\lambda}{2} v_s^\mathsf{H} R_{XX}^{-1} v_s &= 1
	\\
	 - \frac{\lambda}{2} &= (v_s^\mathsf{H} R_{XX}^{-1} v_s)^{-1}
\end{align*}

\begin{align*}
	w_{mvdr} &= (v_s^\mathsf{H} R_{XX}^{-1} v_s)^{-1} R_{XX}^{-1} v_s 
\end{align*}

OLD
\begin{align*}
	w_{mvdr} &= (v_s^\mathsf{H} R_{XX}^{-1} v_s)^{-1} R_{XX}^{-1} v_s 
	\\
	\mathcal{P} &= \frac{1}{T} \sum_{t=0}^{T-1} | w^\mathsf{H} x(t) | ^2
	\\
	&= \frac{1}{T} \sum_{t=0}^{T-1} | (v_s^\mathsf{H} R_{XX}^{-1} v_s)^{-1} v_s ^\mathsf{H} R_{XX}^{-1}  x(t) | ^2
	\\
	&= \left| (v_s^\mathsf{H} R_{XX}^{-1} v_s)^{-1} \right| ^ 2 (v_s ^\mathsf{H} R_{XX}^{-1} ) \frac{1}{T} \sum_{t=0}^{T-1}  | x(t) | ^2 (R_{XX}^{-1}v_s)
	\\
	&= \left| (v_s^\mathsf{H} R_{XX}^{-1} v_s)^{-1} \right| ^ 2 ( v_s ^\mathsf{H} R_{XX}^{-1} ) R_{XX} (R_{XX}^{-1}v_s)
	\\
	&= \left| (v_s^\mathsf{H} R_{XX}^{-1} v_s)^{-1} \right| ^ 2  v_s ^\mathsf{H} R_{XX}^{-1}v_s
	\\
	\mathcal{P}_{mvdr} &= (v_s^\mathsf{H} R_{XX}^{-1} v_s)^{-1} 
\end{align*}

NEW
\begin{align*}
	w_{mvdr} &= (v_s^\mathsf{H} R_{XX}^{-1} v_s)^{-1} R_{XX}^{-1} v_s 
	\\
	\mathcal{P} &= \frac{1}{T} \sum_{t=0}^{T-1} | w^\mathsf{H} x(t) | ^2
	\\
	&= \frac{1}{T} \sum_{t=0}^{T-1} | (v_s^\mathsf{H} R_{XX}^{-1} v_s)^{-1} v_s ^\mathsf{H} R_{XX}^{-1}  x(t) | ^2
	\\
	&= \left| (v_s^\mathsf{H} R_{XX}^{-1} v_s)^{-1} \right| ^ 2 \frac{1}{T} \sum_{t=0}^{T-1} | ( v_s ^\mathsf{H} R_{XX}^{-1}  x(t) | ^2
	\\
	&= \left| (v_s^\mathsf{H} R_{XX}^{-1} v_s)^{-1} \right| ^ 2 ( v_s ^\mathsf{H} R_{XX}^{-1} R_{XX} | ^2
\end{align*}

NEW
\begin{align*}
	\mathcal{P}_{mvdr} &= (v_s^\mathsf{H} R_{XX}^{-1} v_s)^{-1} 
\end{align*}

Forward Backward Averaging
\begin{align*}
	R_{XX, fb} &= \frac{1}{2} \left( R_{XX} + J R_{XX}^* J  \right)
\end{align*}

Diagonal Loading
\begin{align*}
	R_{XX, dl} &= R_{XX} + \sigma_L I
\end{align*}


\end{document}
